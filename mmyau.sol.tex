\begin{solutionseries}[]{Chapter\unhbox \voidb@x \protect \penalty \@M \ \protect \let \reserved@d = *\def \par 
\begin{exsol@solution}{1.1.1}
	\begin{multicols}{3}
	\begin{enumerate}[label = (\alph*)]
		\item I'm too lazy to compute these rn
	\end{enumerate}
	\end{multicols}
	Note that they do not equal.
\end{exsol@solution}
\begin{exsol@solution}{1.1.2}
	Too lazy, but follows from commutation in reals.
\end{exsol@solution}
\begin{exsol@solution}{1.1.3}
	\begin{enumerate}[label = (\alph*)]
		\item Let \(z=a+ib\). Then, \(z\ast = a-ib\) and \((z\ast)\ast = a-(-ib) = a+ib = z\).
		\item \begin{enumerate}[label = (\roman*)]
			\item too lazy
		\end{enumerate}
	\end{enumerate}
\end{exsol@solution}
\begin{exsol@solution}{1.2.1}
	\[e^{-i\theta} = \boxed{\cos\theta-i\sin\theta}\]
	Thus, they are \(\boxed{\text{conjugates}}\).
\end{exsol@solution}
\begin{exsol@solution}{1.3.1}
	\[\Re(z) = \frac{z+z\ast}{2}\qquad\qquad \Im(z) = \frac{z-z\ast}{2}\]
\end{exsol@solution}
\begin{exsol@solution}{1.3.2}
	\begin{enumerate}[label = (\alph*)]
		\item Let \(z = a+ib\) where \(a,b\in\R\). Substituting and cancelling,
			\[z=z\ast\]
			\[a+ib = a-ib\]
			\[ib = -ib\]
			\[b = -b\]
			\[b=0\]
			thus,
			\[z = a\in\R\]
		\item We have
			\[(z\ast z)\ast = (z\ast)\ast z\ast = zz\ast = z\ast z\]
			by commutivity. Thus, since \(z\ast z\) is its on conjugate, it must be purely real.
	\end{enumerate}
\end{exsol@solution}
\begin{exsol@solution}{1.3.3}
	\[e^{i\theta} = \cos\theta + i\sin\theta\]
\end{exsol@solution}
\begin{exsol@solution}{1.3.4}
	\begin{enumerate}[label = (\alph*)]
		\item too lazy		
	\end{enumerate}
\end{exsol@solution}
\begin{exsol@solution}{1.3.5}
	\begin{multicols}{2}
	\begin{enumerate}[label = (\alph*)]
		\item i don't want to
	\end{enumerate}
	\end{multicols}
\end{exsol@solution}
\begin{exsol@solution}{1.3.6}
	\(\boxed{1,i,-1,-i}\)	
\end{exsol@solution}
\begin{exsol@solution}{1.3.7}
	\(\boxed{1, \frac{1}{2}+i\frac{\sqrt{3}}{2}, -\frac{1}{2}+i\frac{\sqrt{3}}{2},-1,-\frac{1}{2}-i\frac{\sqrt{3}}{2},\frac{1}{2}-i\frac{\sqrt{3}}{2}}\)
\end{exsol@solution}
\end{solutionseries}
\begin{exsol@solution}{2.1.1}
	literally just do what I said
\end{exsol@solution}
\begin{exsol@solution}{2.1.2}
	again, just do what I said.
\end{exsol@solution}
