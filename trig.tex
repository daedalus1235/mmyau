%! TEX root = mmyau.tex
\chapter{Trigonometric Functions}\label{ch:trig}
\section{Trig Identities}
Pretty much all useful trig identities can be derived from just Euler's formula
\begin{equation}
	e^{i\theta} = \cos\theta + i\sin\theta \tag{\ref{eq1:euler}}
\end{equation}
and all of the familiar properties of the exponential function. Firstly, we see that we can write\footnote{Or, in fact, \emph{define}}
\begin{equation}
	\cos\theta = \frac{e^{i\theta}+e^{-i\theta}}{2}\qquad\qquad\sin\theta\frac{e^{i\theta}-e^{-i\theta}}{2i}
\end{equation}
As seen in problem 1.2.1, we have
\[e^{-i\theta} = \cos\theta - i\sin\theta\]
so
\[e^{-i\theta} = \left(e^{i\theta}\right)\ast\]
thus, we have that 
\[\abs{e^{i\theta}}=\sqrt{e^{i\theta}e^{-i\theta}} = \sqrt{e^{0}} = 1\]
and we obtain the Pythagorean identity
\begin{equation}
	\abs{e^{i\theta}}^2 = \cos^2\theta+\sin^2\theta = 1
\end{equation}

Additionally, using the fact that
\[e^{-i\theta} = e^{i(-\theta)}\]
we can find the parity (evenness or oddness) of sine and cosine:
\begin{equation}
	\cos(-\theta) = \cos\theta \qquad\qquad \sin(-\theta) =-\sin\theta
\end{equation}
By multiplying the Euler formula,
\[e^{i(\theta+\phi)} = e^{i\theta}e^{i\phi}\]
and matching coefficients, (i.e., setting real parts equal and imaginary parts equal), we can obtain the angle addition formulae
\begin{subequations}
	\begin{equation}
		\sin(\theta+\phi) = \sin\theta\cos\phi + \cos\theta\sin\phi
	\end{equation}
	\begin{equation}
		\cos(\theta+\phi) = \cos\theta\cos\phi -\sin\theta\sin\phi
	\end{equation}
\end{subequations}
Similarly, we can find the angle subtraction formulae, either by multiplying \(e^{i(\theta-\phi)} = e^{i\theta}e^{-i\phi}\), or by using the parity of sine and cosine. 
\begin{exercise}
	Do just that---derive the angle subtraction formulae
\begin{solution}
	literally just do what I said
\end{solution}
\end{exercise}

Combining the angle addition formulae with the pythagorean identity, we obtain the angle reduction formulae
\begin{subequations}
	\begin{align}
		\sin2\theta &= 2\sin\theta\cos\theta\\
		\cos2\theta &= \frac{1}{2}\cos^2\theta - \frac{1}{2}\\
			    &=\frac{1}{2}-\frac{1}{2}\sin^2\theta
	\end{align}
\end{subequations}
\begin{exercise}
	Verify the above are true.
\begin{solution}
	again, just do what I said.
\end{solution}
\end{exercise}

\subsection{de Moivre's Theorem}
de Moivre's theorem relies on a very basic premise:
\[(e^{i\theta}) = e^{in\theta} = e^{i(n\theta)}\]
Using Euler's formula, we can expand the leftmost and rightmost terms to obtain de Moivre's theorem:
\begin{equation}
	\left(\cos\theta + i\sin\theta\right)^n = \cos n \theta + i\sin n\theta
\end{equation}
From de Moivre's theorem, we can obtain angle reduction formulae by expanding the LHS and 

\section{Hyperbolic Trig Functions}


\section{Substitutions}
I legitimately cannot remember what examples to use here but the only thing i can think of are boosts

