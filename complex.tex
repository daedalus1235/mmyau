%! TEX root = mmyau.tex
\chapter{Complex Numbers}
\section{The Imaginary Unit}
As you may recall, there is no real solution to the polynomial 
\[x^2+1=0\]
Rather, we define a new number, the imaginary unit \(i\), such that
\[i^2\equiv-1\]
This is a number just like any real number, but doesn't add directly to real numbers. Instead, when added to a real number, it forms a \emph{complex number}. The set of all complex numbers is denoted \(\C\). Examples of complex numbers are: \(1+2i, 4, 3i, \pi+i\sqrt{e}\).

In general arithmetic with complex numbers is very similar to arithmetic of algebraic variables, albeit with one caveat: we replace any occurences of \(i^2\) with \(-1\). For instance, let \(w,z\in\C\) be complex numbers. We may write, for real coefficients \(a,b,c,d\in\R\) that \(z=a+ib, w= c+id\) for suitable coefficients. We can then perform addition as normal:
\[z+w = a+ib + c+id = (a+c) + i(b+d)\]
Similarly, we can multiply using FOIL:\@
\[zw = (a+ib)(c+id) = ac + ibc + iad + i^2bd = (ac-bd) +i(ad+bc)\]

\subsection{The Complex Plane}
A complex number can be separated into its real and imaginary components as follows
\begin{equation}
	z = \Re(z) + i\Im(z)
\end{equation}
where \(\Re(z)\) is called the real part (as it is not attached to the imaginary unit) and \(\Im(z)\) is called the imaginary part (as it is attached to the imaginary unit). For example, if we consider \(z = a+ib\), we have the real part of \(z\) to be \(\Re(z)=a\) and the imaginary part to be \(\Im(z) = b\). Note that while the imaginary part is called the imaginary part, it is a \emph{real} number. 

We can plot the complex number on the 2D plane by using the mapping
\[a+ib \mapsto (a,b)\]
That is, we treat the real component as the \(x\) position and the imaginary component as the \(y\) position.

This visualization lends itself nicely to an interpretation of the size of a complex number. Just as the size of a real number is given by how far it is from 0, the size of a complex number is given by how far it is from \(0+i0 = 0\). We can do this of course using pythagorean theorem:
\begin{equation}
	\abs{z} = \sqrt{\Re(z)^2+\Im(z)^2}
\end{equation}
The magnitude of the complex number \(z=a+ib\) would then be \(\abs z = \sqrt{a^2+b^2}\).

What happens now, if we
\section{Euler's Formula}
Euler's formula is by far the most useful result of complex numbers in computation. It states that
\begin{equation}
	e^{i\theta} = \cos\theta + i\sin\theta \label{eq1:euler}
\end{equation}
We can show that this is true by comparing Taylor Series expansions of both sides. Recall:
\begin{align*}
	\exp(x) &= 1 + x + \frac{1}{2}x^2 + \frac{1}{3!} x^3+\frac{1}{4!}x^4+\frac{1}{5!}x^5\dots\\
	\cos(x) &= 1 - \frac{1}{2}x^2 + \frac{1}{4!}x^4+\dots\\
	\sin(x) &= x - \frac{1}{3!}x^3+\frac{1}{5!}x^5+\dots
\end{align*}
Computing the LHS of Equation~\ref{eq1:euler},
\begin{align*}
	\exp(ix) &= 1 + ix -\frac{1}{2}x^2 - \frac{i}{3!} x^3 + \frac{1}{4!}x^4 + \frac{i}{5!}x^5+\dots\\
	\intertext{Similarly, the terms on the RHS can be computed as}
	\cos(x)&=1\hphantom{+ix}-\frac{1}{2}x^2\hphantom{-\frac{i}{3!}x^3}+\frac{1}{4!}x^4\hphantom{+\frac{i}{5!}x^5}+\dots\\
	i\sin(x)&=\hphantom{1+}ix\hphantom{-\frac{1}{2}x^2}-\frac{i}{3!}x^3\hphantom{+\frac{1}{4!}x^4}+\frac{i}{5!}x^5+\dots
\end{align*}
As you should be able to see, the terms in the Taylor expansions of both sides matches. This equation will be very useful in deriving trigonometric identities, as will be seen in the next chapter, but truly shows its power in quantum mechanics.

\subsection{Polar Complex Numbers}
Euler's formula lends itself nicely to an additional way to denote complex numbers, which is the \emph{polar form}. The easiest way to visualise this is through the \emph{complex plane}. This lets us plot the complex numbers on a 2D cartesian plane. We consider the real part of the complex number to be 


