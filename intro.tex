%! TEX root = mmyau.tex
\chapter*{Introduction}
While physics students are often required to take Math Methods or upper level mathematics coursework, these often focus moreso on the abstract nature and usefulness of advanced math concepts rather than tricks to make computation easier or circumvent their use altogether. This text is a collection of some typical math methods, as well as some particular tricks that I've found useful that weren't emphasized as much in the curriculum I took at CMU, as well as some topics that I find interesting and may be useful. 

Note, that while this book has some semblance of rigorous-ness, it is for, first and foremost, \emph{physicists}, and not mathematicians. Most objects we deal with in physics are nice and well-behaved (except the Dirac Delta, of course), and so, many of the caveats that you'd have to deal with in a typical analysis course will be glossed over or completely ignored. 

This book assumes familiarity with the concepts of multivariate calculus and high-school algebra. It will begin with showing the usefulness of complex numbers and their associated properties, then continue on with the introduction of linear algebra techniques. Finally, it will conclude with a variety of miscellaneous topics, such as orthonormal coordinates and differential equations.
