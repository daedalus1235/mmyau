%! TEX root = mmyau.tex
\chapter{Simplifying Algebra}
\section{Definition and Substitution}
\emph{Never} carry around bulky expressions; chances are you'll mess up copying it down at some point and end up with a fraction that doesn't reduce or a term that doesn't cancel. Rather, you should always define a placeholder variable and carry that around instead.


\section{The Power of a Unit}
\subsection{Strategically Multiplying by One}
\subsection{Strategically Adding Zero}
A similar technique can be accomplished in addition;

\subsubsection{Lagrange Multipliers}
One well known example of strategically adding zero is of course the method of \emph{Lagrange multipliers} in the optimization  of a function subject to a constraint.

\subsection{Strategically using the Identity}
We can extend this idea of strategically multiplying by one to strategically multiplying by the identity.

\subsection{Strategically Differentiating WRT One}
Perhaps one of the most egregious abuse of a unit is to differentiate with respect to the number one. As we can arbitrarily insert a 1 anywhere, we can insert a variable \(\xi \equiv 1\), take the derivative \(\partial_\xi\), and evaluate it at \(\xi = 1\). In fact, this can be used to derive the Gibbs-Duhem relation in thermodynamics. For example, say I wish to calculate
\[\int_0^\infty x^2e^{-x^2}\d{x}\]
Obviously this is a gamma function in disguise (just substitute \(u=x^2\)), but say we don't know that.

\section{Superposition Principle}
Remember that the superposition principle works with both addition \emph{and subtraction}. Sometimes it is easier to consider the lack of an object rather than compute a wider sum or the whole. For example, consider a disc radius \(r\) and mass density \(\rho\), with a hole drilled out of it at \(r/2\) with radius \(r/2\). If we want to calculate the centre of mass, we could 
\[r_{cm} = \]
however, we could instead view it as the sum of the larger disc and a smaller ``anti-disc.'' This turns the calculation from a tedious integral to a simple weighted sum. We can compute the inertia tensor in the same way. [excersize: do this]

Moment of Inertia and CoM of disk with missing disk

